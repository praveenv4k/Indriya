% Chapter Template

\chapter{Conclusion} % Main chapter title
\label{Chapter8} % Change X to a consecutive number; for referencing this chapter elsewhere, use \ref{ChapterX}
\lhead{Chapter 8. \emph{Conclusion}} 
	Human Robot Interaction is a complex interdisciplinary problem that needs expertise in various fields like human computer interaction, psychology, biomechanics, humanoid robotics etc., Social robots have become very common in the recent days and they find useful applications in entertainment, education, autism treatment and elderly care. 
	
	For an efficient interaction with human, the robot has to understand his motions and also the environment around him. This is only possible when the robot has necessary perception capabilities which seemlessly make the robot understand the human motions. However the onboard sensors most often cannot deliver all the information necessary for an effective interaction. So there is a need to augment exteroceptive sensors to the system and RGB-D sensors prove to be a convincing solution to this purpose. Irrespective of the underlying perception system used, there is also a need for a scalable robot behavior design framework which could take care of automatic information flow between various components in the system and dynamic creation of behaviors based on object permanence in the environment. Combining the perception system and behavior framework will result in a platform wherein interaction based on human motion will be possible. 
	
	This bibliographic study explained the requirements and challenges in each of the problems of motion understanding, robot localization, behavioral design and evaluation techniques. The state of the art research helped to make appropriate choices to tackle each of the problems under study. More specifically in the research plan described in this study, it is proposed to use Kinect V2 sensor as the perception system and use of Kinect SDK for human motion understanding. The robot localization and tracking problem will be tackled by using a parallelized implementation of particle filter tracking algorithm available in Point cloud library. The Target Drives Means(TDM) framework will be used as the dynamic behavior design infrastructure. An example TDM program is also presented that uses the aforementioned experimental platform. The state of the art HRI evaluation techniques like self assesments and behavioral measures will be used to evaluate the interaction. The experimental platform this thesis aims to achieve will encourage diverse experiments focusing on the evaluation of specific interaction metrics.