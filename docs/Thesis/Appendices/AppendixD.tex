% Appendix Template

\chapter{User study questionnaire} % Main appendix title

\label{AppendixD} % Change X to a consecutive letter; for referencing this appendix elsewhere, use \ref{AppendixX}

\lhead{Appendix D. \emph{User study questionnaire}} % Change X to a consecutive letter; this is for the header on each page - perhaps a shortened title

A questionnaire is used (English/Japanese) for the user data collection.  
\section*{Questionnaire/アンケート}


\begin{Form}
\hrule
\subsection*{Usability/使いやすさについて}
\hrule
\radioquestion{1}{Was the interface easy to use?}{このインターフェースは使いやすかったですか?}{Very easy}{Very difficult}{簡単だった}{難しかった}

\radioquestion{2}{Do you think tablet application is convenient?}{今回のタブレットアプリケーションは便利でしたか?}{Very easy}{Very difficult}{とてもそう思う}{そうは思わない}

\radioquestion{3}{How long did it take to learn to use the application?}{このアプリケーションに慣れるのにどのくらいかかりましたか?}{Very easy}{Very difficult}{すぐに慣れた}{とても時間がかかった}

\radioquestion{4}{Was the program structure easy to understand?}{プログラムの構成は分かりやすいものでしたか?}{Very easy}{Very difficult}{簡単だった}{難しかった}

\radioquestion{5}{Were the functionality of programming blocks easy to understand?}{プログラムブロックは分かりやすいものでしたか?}{Very easy}{Very difficult}{簡単だった}{難しかった}

\radioquestion{6}{Were you able to create the scenario you wanted to?}{思い通りのシナリオを作ることができましたか?}{Very easy}{Very difficult}{問題なく作ることができた}{全然作れなかった}
\hrule
\subsection*{Execution/シナリオの実行について}
\hrule
\radioquestion{7}{Was the scenario executed as you expected?}{思い通りにシナリオを実行することができましたか?}{Very easy}{Very difficult}{思い通りにできた}{全然違うものだった}

\radioquestion{8}{Did the robot understand your gestures and commands as expected?}{ロボットがあなたのことを理解しているように感じましたか?}{Very easy}{Very difficult}{きちんと理解してくれた}{全然理解してくれなかった}

\radioquestion{9}{Was the robot reactive to your commands and gestures?}{ロボットはあなたの行動に対して適切に反応していましたか?}{Very easy}{Very difficult}{きちんと反応してくれた}{全然反応してくれなかった}

\vspace{.20in}
%\hrule 
\begin{tabular}{rl}
\textbf{10} & \textbf{If the execution has failed, what were the reasons for it?}\\
{} & (You can choose more than one option for this question)\\
{} & \textbf{シナリオの実行がうまくいかなかったとき、その理由が分かりましたか?}\\
{} & (複数回答可))\\
\end{tabular}
%\hrule 
%\vspace{.5em}

\begin{tabular}{rl}
$\square$ & System Failure/システムに原因があった \\
$\square$ & System Failure/システムに原因があった \\
$\square$ & System Failure/システムに原因があった \\
$\square$ & System Failure/システムに原因があった \\
\end{tabular}
\vspace{.05in}


\end{Form}